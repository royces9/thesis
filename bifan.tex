\setlength{\abovedisplayskip}{0pt}
\setlength{\belowdisplayskip}{0pt}
\def \xoneavg{\totalcopy{x1} - \avgcopynum{x1x2} - \avgcopynum{x1y2}}
\def \xonecopy{\totalcopy{x1} - \copynum{x1x2} - \copynum{x1y2}}
\def \xtwoavg{\totalcopy{x2} - \avgcopynum{x1x2} - \avgcopynum{y1x2}}
\def \xtwocopy{\totalcopy{x2} - \copynum{x1x2} - \copynum{y1x2}}
\def \yoneavg{\totalcopy{y1} - \avgcopynum{y1y2} - \avgcopynum{y1x2}}
\def \yonecopy{\totalcopy{y1} - \copynum{y1y2} - \copynum{y1x2}}
\def \ytwoavg{\totalcopy{y2} - \avgcopynum{y1y2} - \avgcopynum{x1y2}}
\def \ytwocopy{\totalcopy{y2} - \copynum{y1y2} - \copynum{x1y2}}

\section*{Bifan Motif}
\addcontentsline{toc}{chapter}{Bifan Motif}

\subsection*{Introduction}
\addcontentsline{toc}{section}{Introduction}
%
The bifan is a simple symmetrical extension of the CB3 motif.
%
By adding another protein, \ce{y1}, that can compete with \ce{x1}, there are now two simultaneous\
pathways for the network to go through.
%
There are now two on-target complexes, \ce{x1x2} and \ce{y1y2}; and two\
off-target complexes, \ce{x1y2} and \ce{y1x2}.
%
Each specificity measures the selectivity for each of the single proteins to be bound,\
in one complex over another and are defined again below:
%
\bigskip
\begin{equation}
  \begin{split}
    S_{\ce{x1}} = \frac{\copynum{x1x2}}{\copynum{x1y2}} = \frac{\conc{x1x2}}{\conc{x1y2}}\\
    S_{\ce{x2}} = \frac{\copynum{x1x2}}{\copynum{y1x2}} = \frac{\conc{x1x2}}{\conc{y1x2}}\\
    S_{\ce{y1}} = \frac{\copynum{y1y2}}{\copynum{y1x2}} = \frac{\conc{y1y2}}{\conc{y1x2}}\\
    S_{\ce{y2}} = \frac{\copynum{y1y2}}{\copynum{x1y2}} = \frac{\conc{y1y2}}{\conc{x1y2}}
  \end{split}
  \bigskip
\end{equation}
%

The bifan motif is known to be overrepresented in network motifs \cite{functionsofbifan, biocircuits}, and is also\
extendable to larger networks with five or more interacting proteins.
%

\subsection*{Deterministic Model}
\addcontentsline{toc}{section}{Deterministic Model}
%
Using a similar treatment to the CB3 motif, the following expressions for steady\
state can be found for the bifan motif:
%
\bigskip
\begin{equation}
  \begin{split}
    & \frac{k_{off_{xx}}}{k_{on}} =\
    \frac{({\conc{x1}}_0 - \conc{x1x2} - \conc{x1y2})({\conc{x2}}_0 - \conc{x1x2} - \conc{y1x2})}{\conc{x1x2}}\\
    & \frac{k_{off_{xy}}}{k_{on}} =\
    \frac{({\conc{x1}}_0 - \conc{x1x2} - \conc{x1y2})({\conc{y2}}_0 - \conc{y1y2} - \conc{x1y2})}{\conc{x1y2}}\\
    & \frac{k_{off_{yy}}}{k_{on}} =\
    \frac{({\conc{y1}}_0 - \conc{y1y2} - \conc{y1x2})({\conc{y2}}_0 - \conc{y1y2} - \conc{x1y2})}{\conc{y1y2}}\\
    & \frac{k_{off_{yx}}}{k_{on}} =\
    \frac{({\conc{y1}}_0 - \conc{y1y2} - \conc{y1x2})({\conc{x2}}_0 - \conc{x1x2} - \conc{y1x2})}{\conc{y1x2}}\\
  \end{split}
  \label{ss_bifan}
\end{equation}  
%


\subsection*{Stochastic Model}
\addcontentsline{toc}{section}{Stochastic Model}
%
Much like the CB3 motif, probabilities for each of the species of the\
motif can be defined as follows with respect to the complexes:
%
\smallskip
\begin{equation}
  \begin{split}
    & P_{\ce{x1}} = \poissmore{{\xoneavg}}{{\xonecopy}}\\
    & P_{\ce{x2}} = \poissmore{{\xtwoavg}}{{\xtwocopy}}\\
    & P_{\ce{y1}} = \poissmore{{\yoneavg}}{{\yonecopy}}\\
    & P_{\ce{y2}} = \poissmore{{\ytwoavg}}{{\ytwocopy}}\\
    & P_{\ce{x1x2}} = \poissonexpr{\ce{x1x2}}\\
    & P_{\ce{x1y2}} = \poissonexpr{\ce{x1y2}}\\
    & P_{\ce{y1y2}} = \poissonexpr{\ce{y1y2}}\\
    & P_{\ce{y1x2}} = \poissonexpr{\ce{y1x2}}
  \end{split}
  \label{bifan_individual_prob}
\end{equation}
%

Replacing the probabilities from Eq. \ref{bifan_individual_prob}\
into Eq. \ref{bifan_total_prob}, the 8-dimensional probability is\
reduced to a 4-dimensional probability dependent only on\
the complexes and the initial starting amount of lone proteins.
%
\bigskip
\begin{equation}
  P(\copynum{x1x2}, \copynum{x1y2}, \copynum{y1y2}, \copynum{y1x2}) = \
  P_{\ce{x1}} *P_{\ce{x2}} *P_{\ce{y1}} *P_{\ce{y2}}\
  *P_{\ce{x1x2}} *P_{\ce{x1y2}} *P_{\ce{y1y2}} *P_{\ce{y1x2}}
  \label{bifan_total_prob}
  \bigskip
\end{equation}
%

Because of the factorial dependence in the expression for the Poisson\
distribution, the calculation for the distribution becomes very taxing\
as the system size is increased, for similar reasons, the chemical\
master equation also becomes cumbersome to use.
%
But the Poisson can be assumed to be approximated by a Gaussian distribution,\
for large systems, then the specificity can be approximated using\
Marsaglia's approximation.
%

However, Marsaglia's approximation to the specificity can not be applied\
to the bifan motif because finding the standard deviation of the system\
transformed to two variables proves to be difficult.
%
Because Marsaglia's approximation is unfit for the bifan motif\
we will use simulation data to study the specificity.
%
And like in the CB3 motif, we will change any of the zero denominators\
to one, to avoid any discontinuities.


\subsection*{Results}
\addcontentsline{toc}{section}{Results}

\subsubsection*{The deterministic specificity's are monotonic functions of protein concentration}
%\subsubsection*{The deterministic solution of the specificity for the bifan motif can be calcuated numerically}
From Eq. \ref{ss_bifan}, the deterministic specificity can be solved as a\
function of $\copynum{x1}$, and is plotted in Fig. \ref{deterministic_bifan}.
%

\begin{figure}[H]
  \includegraphics[width=\textwidth]{figures/bifan/deterministic/deterministic_bifan_conc.png}
  \caption{Plot of each of the four specificities of bifan as a function of ${\conc{x1}}_0$,
    ${\conc{x2}}_0 = {\conc{y1}}_0 = {\conc{y2}}_0 = 0.1 \mu M$, V = 1}
  \label{deterministic_bifan}
\end{figure}

Both \spec{x1} and \spec{y2} behave qualitatively similarly to the CB3 specificity, and are\
decreasing functions of $\totalcopy{x1}$.
%
\spec{x2} and \spec{y1} both appear to grow without bound in the deterministic model, and are\
increasing functions of $\totalcopy{x1}$.


\subsubsection*{Changing zero denominators to one for the bifan motif gives results consistent with the raw data}
%
The bifan also runs into issues with the denominator becoming infinite,\
and like the CB3 motif, any time the denominator is zero, we change the zero\
denominator to one to make the calculations tractable.
%

\begin{figure}[H]
  \includegraphics[width=\textwidth]{figures/bifan/raw_plot/raw_plot.png}
  \caption{Plot of the simulation specificities against $\copynum{x1}$,\
    $\totalcopy{x2} = \totalcopy{y1} = \totalcopy{y2} = 100$, V = 1}
  \label{bifan_raw_plot}
\end{figure}

Without changing the denominators, the behaviour of the specificity in\
simulation is as seen in Fig. \ref{bifan_raw_plot}.
%
When the zero denominators are changed to one, we obtain Fig. \ref{bifan_fixed_plot}.
%The mean with an error bar of $\pm \sigma$ is plotted in Fig. \ref{bifan_fixed_plot}.
%
When comparing Fig. \ref{bifan_fixed_plot} to Fig. \ref{bifan_raw_plot},\
the general behaviour of the \perc{10}, \perc{90}, and median are the same.
%
Because the general behaviour is the same, for the rest of the analysis,\
the deviation to the denominator will be added much like in the CB3 motif.
%
It is important to note that changing the denominator does have a\
non-negligible effect on the specificity, for values\
of $\copynum{x1}$ at very low and high copy numbers.
%

\begin{figure}[H]
  \includegraphics[width=\textwidth]{figures/bifan/specificity/spec_simulation_fixed.png}
  \caption{Plot of the simulation specificities with zero denominators changed to one\
    against $\copynum{x1}$,\
    $\totalcopy{x2} = \totalcopy{y1} = \totalcopy{y2} = 100$, V = 1}
  \label{bifan_fixed_plot}
\end{figure}
%

%\subsubsection*{The median specificity and the width of the distribution are correlated}
%\subsubsection*{The specificity exhibits similar behaviour to the CB3 motif's specificity}
Like the CB3 motif, the distribution of the specificity widens with the mean.
%
Furthermore, the general behaviour of \spec{x1} and \spec{y2} are both similar\
to the behaviour of CB3 motif specificity, such as being a decreasing\
function of $\totalcopy{x1}$.
%
However, only \spec{x1} displays non-monotonic behaviour\
at very low $\totalcopy{x1}$.
%
\spec{x2} and \spec{y1}, in contrast to the other two specificities,\
are both increasing functions of $\totalcopy{x1}$.
%
Also similar to the CB3 specificity, the width of the\
distribution grows as the mean of the specificity increases.
%


%% \begin{figure}[H]
%%   \includegraphics[width=\textwidth]{figures/bifan/specificity/spec_blue_differentQ.png}
%%   \caption{Means of specificities of bifan with an error bar of $\pm \sigma$,\
%%     $\totalcopy{x2} = \totalcopy{y1} = \totalcopy{y2} = 100$, V = 1}
%%   \label{bifan_mean_sigma}
%% \end{figure}


\subsubsection*{The specificity is a function of the complex copy numbers}
Similarly to the CB3 motif, the behaviour of the specifities can be\
explained by looking at the individual complexes.
%
The wide distributions of \spec{x2} and \spec{y1} are due to the\
low amount of \ce{y1x2}, the denominator, compared to the respective\
on-target complex, the numerator.
%
Likewise, for \spec{x1} and \spec{y2}, the amount of \ce{x1y2}\
increases as a function of $\totalcopy{x1}$.
%
The behaviour of the complexes themselves also garner interest.
%

\begin{figure}[H]
  \includegraphics[width=\textwidth]{figures/bifan/complex/complex_error.png}
  \caption{Means of complexes of bifan with an error bar of $\pm \sigma$,\
    $\totalcopy{x2} = \totalcopy{y1} = \totalcopy{y2} = 100$, V = 1}
  \label{bifan_complex}
\end{figure}

Except for \ce{y1y2}, each of the complexes is monotonic with respect\
to $\totalcopy{x1}$, and each complex crosses another at 100 copy numbers of\
\ce{x1}.
%
The crossing at 100 copy number is significant because it is the point\
where the minimum valued complex is maximized, it is also\
where the initial conditions of the system are all symmetrical.
%

%think of a better title
\subsubsection*{The minimum noise in the system reaches a maximum at the same point as the mean of the specificities}
The Fano factor, defined as $\sigma ^2 / \mu$, of the simulation data was\
plotted as a function of $\totalcopy{x1}$, (Fig. \ref{bifan_fano}).
%
In addition to the minimum valued mean being maximized at this point, the minimum\
valued Fano factor is also maximized at the same point.
%
Which is to say, this point is where the distributions of the complexes\
collectively is the largest compared to the mean.

\begin{figure}[H]
  \includegraphics[width=\textwidth]{figures/bifan/fano/fano.png}
  \caption{Plot of the Fano factor vs. $\copynum{x1}$,\
    $\totalcopy{x2} = \totalcopy{y1} = \totalcopy{y2}$ = 100, V = 1}
  \label{bifan_fano}
\end{figure}

The product of the Fano factors for the complexes also reaches\
a maximum at the same point, which reiterates that the noise is maximized\
when the system is in symmetry.
%


%%% Local Variables: ***
%%% mode: latex ***
%%% TeX-master: "thesis.tex" ***
%%% End: ***
