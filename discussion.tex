\section*{Discussion}
\addcontentsline{toc}{chapter}{Discussion}
%
We find that width of the distribution for the specificity in both\
the CB3 and bifan motif increases with the mean of the specificity.
%
For the CB3 motif, the deterministic specificity always decreases as a\
function of concentration, suggesting that there are two conflicting needs\
balanced by the cell: increased specificity (at lower concentrations) and\
increased information transfer in terms of molecular targets (at higher concentrations).
%
The stochastic specificity adds another consideration to the picture, where the cell\
balances high average specificity with high precision in specificity.
%
The precision, or the magnitude of fluctuations in specificity, are non-monotonic with\
respect to protein copy number, which suggests another way for the cell to optimize\
relative protein concentrations.
%
%In general, increases\
%as $\totalcopy{x1}$ is decreased, but decreases again for very small\
%$\totalcopy{x1}$; this non-monotonicity is also seen in \ce{x1x2}'s\
%Fano factor.
%
In the bifan motif, the stochastic \spec{y1} and \spec{x2} exhibit\
opposite behaviour from the CB3 motif, displaying increasing\
behaviour as $\totalcopy{x1}$ is increased, but for \spec{y1} as\
$\totalcopy{x1}$ is increased more, the specificity decreases.
%
\spec{x1} follows the same trends as the CB3 specificity,\
with the same non-monotonic behaviour; \spec{y2} follows\
the same shape as the CB3 specificity but does not display\
a down turn at very low $\totalcopy{x1}$ copy number.
%

This non-monotocity suggests a certain tradeoff; if we assume\
higher specificity is better for the cell because it increases information flow through a targeted pathway, the higher median specificity\
may be beneficial to the cell. However, this increased specificity comes with increasing variance,\
which may induce unwanted behaviour, either because the\
fluctuations themselves are disruptive for information flow, or because the higher fluctuations\
give rise to transient entry into lower-than-average specificity states, which could lead to spurious signaling.
%
On the other hand, lower specificity may be less beneficial to the cell,\
but the fluctuations of the specificity are much lower, and so there is\
much better control over the exact value of the specificity.
%
In addition, it has been shown that gene overexpression can be detrimental
to the cell \cite{overexpression}, which suggests the cell may want to\
avoid very wide distributions because the probability of being in a state\
where a specific protein or gene is overexpressed is much higher, however,\
this may come at the cost of not having a specificity high enough for the\
protein to do any meaningful work. [I GOT A LITTLE CONFUSED HERE--if overexpression is detrimental, doesn't that mean the cell prefers wide distributions in specificity?]
%

\newcommand{\discone}{There exist different optimal parameters depending on what the cell requires}
\subsubsection*{\discone}
\addcontentsline{toc}{section}{\discone}
%
As discussed earlier, there are multiple different points of local maxima\
within the CB3 motif, for example, from the parameter set in Fig. \ref{add_one},\
the Gaussian approximation of the median reachs a maximum at about 20 copy numbers\
of $\totalcopy{x1}$.
%
But this is not the only optimization criteria that we can look at for the CB3 motif.
%
A cell, for example, may want to maximize the \perc{10} percentile of the\
specificity distribution.
%
In this scenario, with the parameter set from Fig. \ref{add_one}, the \perc{10}\
percentile can be seen to be maximized at around ~30 copy number of $\totalcopy{x1}$.
%
Alternatively, if the width of the distribution needs to be minimized,\
$\totalcopy{x1}$ would need to be maximized.
%
More generally, the stochastic specificity is interesting in that there\
are many different points that can be optimized for some benefit to the cell.
%
This comes in contrast with the deterministic specificity, which is a strictly decreasing\
function of $\totalcopy{x1}$; maximizing specificity is simply a matter of decreasing the amount of\
$\totalcopy{x1}$.
%

The bifan also shows similar flexibility in the criteria for optimization criteria.
%
In addition to the points brought up for the CB3 motif, which can be applied to\
$\totalcopy{x1}$, each of the other three specificities can also have their own\
maximization criteria, for example \spec{y1} can be seen to have a local\
maximum in the mean at ~500 $\totalcopy{x1}$ copy number in Fig. \ref{bifan_mean_sigma}.
%
Not seen in the CB3 motif is the ability for the bifan motif to optimize the system\
for some or all of the specificities.
%
For example, in the same system, if the system were to attempt to maximumize\
both \spec{x1} and \spec{y1}, the amount of $\totalcopy{x1}$ would be around ~100-200\
copy number.
%


\newcommand{\disctwo}{The behaviour of the specificity is due to the complex's distribution}
\subsubsection*{\disctwo}
\addcontentsline{toc}{section}{\disctwo}
%
%how does the complex's distribution/behaviour relate to the specificity's
%and how does that relate to the cell/biology
From Fig. \ref{bifan_complex} and \ref{cb3_complex_error}, we show that the\
specificity's distribution can be attributed to the values and distributions\
of the individual complexes that make up the specificity.
%
The exact distributions of the complexes seem to be less important than\
the simple fact that the distributions even exist in the first place.
%
Much of the variation in the specificity seems to rely more on the\
denominator being much smaller, and thus, having small deviations\
cause a larger effect on the ratio than the actual width of the\
distributions of the complexes themselves.
%
However, the distributions of the complexes themselves are also of
interest.
%

In Fig. \ref{bifan_complex}, we can see that the minimum value of all of the complexes\
is maximized when the initial conditions are all balanced, in other words, at this point\
is when the system will have the maximal number of each of the complexes.
%
Also at the same point, the product of the Fano factors of each of the complexes is also\
maximized, which tells us that even for the complexes as a whole, more copies leads to\
more noise in the system.
%

However, if we look at each individual complex, that is not the case.
%
For \ce{x1x2}, the amount of copies increases to near saturation,\
but the Fano factor is very small.
%
This is due to $\totalcopy{x1}$ being in so much excess that it\
will push the equilibrium towards creating more \ce{x1x2},\
until it exhausts all of the free \ce{x2} protein.
%
The same can be said for \ce{x1y2}, however, the effect\
is less pronounced because the affinity of \ce{x1} to \ce{y2}\
is weaker than \ce{y1}'s affinity, so there will be some\
competition for the \ce{y2} protein from both \ce{x1} and \ce{y1}.
%


\newcommand{\discthree}{For stochastic analysis, the specificity is an imperfect measure of the system's selectivity}
\subsubsection*{\discthree}
\addcontentsline{toc}{section}{\discthree}
%
The specificity measures the\
system's preference for one pathway over another.
And for deterministic analysis, the continuous nature of variables, lends well to
the use of a ratio.
%
However, for stochastic analysis at small number regimes, the ratio is troublesome\
because the copy numbers of some elements in the system can reach zero,\
which results in infinite/NaN values.
%
This is not to say that the specificity is wrong; it has been shown that\
the specificity is a useful representation of systems\
(\cite{ultrasensitive, komarova, mathematical}) and from\
Fig. \ref{marsaglia_good} and \ref{add_one}, the specificity\
calculated from simulation data and the normal approximation both give\
similar results when not in a low copy number regime, which shows that\
the behaviour we see is real, but we may be missing some part of the\
story by changing zero denominators to one.
%
This may be the case because by doing so, we are basically treating\
the system as being in the same state whether it has one or zero\
copies of the off-target complexes.
%

The region that is undefined for the CB3 motif is fairly small, from Fig.\
\ref{cb3_raw_spec}, the \perc{90} percentile is defined for \ce{x1}\
copy numbers greater than 20, the median is even lower, at approximately 10\
copy number.
%
The specificity works well for the CB3 motif, however, the bifan motif is more\
complicated because of the addition of three more specificities.
%
From Fig. \ref{bifan_raw_plot}, the \perc{90} percentile of \spec{x1} and \spec{y2}\
is defined for values larger than approximately 50 copy numbers of $\copynum{x1}$, but\
\spec{y1} and \spec{x2} is not defined for $\copynum{x1}$ larger than 250.
%
This gives a small range of 50 to 250 copy numbers of $\copynum{x1}$ where all of\
the specifities are both defined.
%
We can see that there is a large region where the specificity is not defined,\
so another measure of the system's selectivity must be used to correctly\
quantify regions where the ratio is undefined.
%

\subsubsection*{Future Work}
%addcontentsline{toc}{section}{Future Work}
%
For the analysis of the system, we studied the steady state distribution of\
the specificity in a closed system with no other reactions happening in the\
volume of our system.
%
It would be beneficial for future study to construct another variable to replace\
the specificity, because the specificity is difficult to handle with low copy\
numbers there may be another measure that can give insight into regimes where\
there specificity is either not defined, or must be altered to be defined.
%
In addition, we know that various insulation mechanisms exist to promote network\
specificity (\cite{ultrasensitive, komarova, mathematical}), and that these will\
also have a non-negligible effect on the system's selectivity.
%
Other effects, such as having an open system, may induce interesting behaviour not\
seen in closed systems, such as multi-stability.


%%% Local Variables: ***
%%% mode: latex ***
%%% TeX-master: ``thesis.tex'' ***
%%% End: *** 
