\def \xoneavg{\totalcopy{x1} - \avgcopynum{x1x2} - \avgcopynum{x1y2}}
\def \xonecopy{\totalcopy{x1} - \copynum{x1x2} - \copynum{x1y2}}
\def \xtwoavg{\totalcopy{x2} - \avgcopynum{x1x2}}
\def \xtwocopy{\totalcopy{x2} - \copynum{x1x2}}
\def \ytwoavg{\totalcopy{y2} - \avgcopynum{x1y2}}
\def \ytwocopy{\totalcopy{y2} - \copynum{x1y2}}

%all of the figure label copy numbers are actual copy numbers
%not copy/volume counts or something weird like that

\setlength{\abovedisplayskip}{0pt}
\setlength{\belowdisplayskip}{0pt}

\section*{CB3 Motif}
\addcontentsline{toc}{chapter}{CB3 Motif}
%
\subsection*{Introduction}
\addcontentsline{toc}{section}{Introduction}
The CB3 system is a simple three element system with a single protein that can bind with one of two\
potential targets to make either one on-target, or one off-target complex.
%
This leads to a specificity, defined again below:
%
\bigskip
\begin{equation*}
  S_x = \frac{\copynum{x1x2}}{\copynum{x1y2}} = \frac{\conc{x1x2}}{\conc{x1y2}}
\end{equation*}
%

The specificity is a measure of how controlled the protein \ce{x1} distributes between\
on and off-target complexes, with higher values corresponding to more of the on-target\
complex compared to off-target complex, and vice versa for lower values of specificity.
%

\subsection*{Deterministic Model}
\addcontentsline{toc}{section}{Deterministic Model}
Assume the system is closed and the system starts with zero copies of the complexes (i.e., all molecular species are initialized as monomers). Define the initial copy number of monomers $x_1$ as $N_{x10}$ (and similar for $x_2$, $y_2$).\
Then the total number of all molecular species in the system containing $x_1$ (in monomers or dimers) will always equal $N_{x10}$ at anytime $t$:
%
\smallskip
\begin{equation}
  \begin{split}
    & \totalcopy{x1} = \copynum{x1} + \copynum{x1x2} + \copynum{x1y2} \\
    & \totalcopy{x2} = \copynum{x2} + \copynum{x1x2} \\
    & \totalcopy{y2} = \copynum{y2} + \copynum{x1y2}
  \end{split}
  \label{totals_copy}
\end{equation}
%

if Eq. \ref{totals_copy} is rearranged for the protein at anytime, we get:
%
\smallskip
\begin{equation}
  \begin{split}
    & \copynum{x1} = \totalcopy{x1} + \copynum{x1x2} +\copynum{x1y2}\\
    & \copynum{x2} = \totalcopy{x2} + \copynum{x1x2} \\
    & \copynum{y2} = \totalcopy{y2} + \copynum{x1y2}
  \end{split}
  \label{totals_copy_rearrange}
\end{equation}
%
``
Dividing Eq. \ref{totals_copy_rearrange} by the volume gives:
%
\smallskip
\begin{equation}
  \begin{split}
    & \conc{x1} = {\conc{x1}}_{0} - \conc{x1x2} - \conc{x1y2} \\
    & \conc{x2} = {\conc{x2}}_{0} - \conc{x1x2} \\
    & \conc{y2} = {\conc{y2}}_{0} - \conc{x1y2} \\
  \end{split}
  \label{totals_conc}
\end{equation}
%

Substituting Eq. \ref{totals_conc} into the the last two equations in Eq. \ref{cb3_de}\
reduces the equation's from all fives species to just the two complexes:
%
\smallskip
\begin{equation}
  \begin{split}
    & \frac{d\conc{x1x2}}{dt} =\
    k_{on}({\conc{x1}}_0-\conc{x1x2}-\conc{x1y2})({\conc{x2}}_0-\conc{x1x2}) -\
    k_{off_{xx}}\conc{x1x2}\\
    & \frac{d\conc{x1y2}}{dt} =\
    k_{on}({\conc{x1}}_0-\conc{x1x2}-\conc{x1y2})({\conc{y2}}_0-\conc{x1y2}) -\
    k_{off_{xy}}\conc{x1y2}
    \label{cb3_non_steady}
  \end{split}
\end{equation}
%

At steady state $\frac{d\conc{a}}{dt} = 0$,
which gives Eq. \ref{ss_cb3}.
%
\bigskip
\begin{equation}
  \begin{split}
    \frac{k_{off_{xx}}}{k_{on}} =\
    \frac{({\conc{x1}}_0-\conc{x1x2}-\conc{x1y2})({\conc{x2}}_0-\conc{x1x2})}{\conc{x1x2}}\\
    \frac{k_{off_{xy}}}{k_{on}} =\
    \frac{({\conc{x1}}_0-\conc{x1x2}-\conc{x1y2})({\conc{y2}}_0-\conc{x1y2})}{\conc{x1y2}}
    \label{ss_cb3}
  \end{split}
\end{equation}
%

\subsection*{Stochastic Model}
\addcontentsline{toc}{section}{Stochastic Model}
%
In a closed chemical reaction system following detailed balance,\
it can be shown that the distribution of any given species\
in that system follows a Poisson distribution \cite{gardiner}.
%
For a reversible reaction $A$, we define $P_a$ as the stoichiometric\
coefficients of the left side of the reaction for specie $a$, and $Q_b$ as\
the stoichiometric coefficient of the right side of the reaction\
for specie $b$.
%
We also denote k$^+_A$ as the stochastic rate constant for the forward reaction,\
and k$^-_A$ as the stochastic rate constant for the backward reaction.
%
With these definitions, the following is shown to be true \cite{gardiner}:
%
\smallskip
\begin{equation}
  k^+_A \prod_a{\avgcopynum{a}^{P_a}}  = k^-_A \prod_b{\avgcopynum{b}^{Q_b}}
  \label{matched_constants}
\end{equation}
%

%With $\avgcopynum{a}$ and $\avgcopynum{b}$ denoting average copy number.
For our system, Eq. \ref{matched_constants} can be simplified to the following.
%
\smallskip
\begin{equation}
  \frac {k^-_{xx}} {k^+} = \frac {\avgcopynum{x1} * \avgcopynum{x2}}{\avgcopynum{x1x2}}\\
  \frac {k^-_{xy}} {k^+} = \frac {\avgcopynum{x1} * \avgcopynum{y2}}{\avgcopynum{x1y2}}
\end{equation}
%

Applying Eq. \ref{totals_copy_rearrange}, \ref{conversion} and \ref{conversion_rate},\
we find the result is exactly the deterministic steady state expression, Eq. \ref{ss_cb3}.
%
%% When the stochastic rate constants and copy number averages are converted\
%% from a copy number basis to a concentration basis, the result\
%% is exactly the deterministic steady state expression, Eq. \ref{ss_cb3}.
%
The average copy number in a stochastic system will follow mass action\
kinetics at steady state.
%

Using the same assumptions, the probability of being in a state $N_A$, for all species $A$,\
is given by a multivariate Poisson (Eq. \ref{total_probability}) \cite{gardiner}.
%
\smallskip
\begin{equation}
  \begin{split}
    P(\copynum{x1}, \copynum{x2}, \copynum{y2}, \copynum{x1x2}, \copynum{x1y2}) = &
    \poissonexpr{\ce{x1}}
    \poissonexpr{\ce{x2}}
    \poissonexpr{\ce{y2}} \\
    & \poissonexpr{\ce{x1x2}}
    \poissonexpr{\ce{x1y2}}
    \label{total_probability}
  \end{split}
\end{equation}
%

Substituting Eq. \ref{totals_copy_rearrange} into Eq. \ref{total_probability}:
%
\smallskip
\begin{equation}
  \begin{split}
    &P(\copynum{x1x2}, \copynum{x1y2}) =\\
    & \poissmore{{\xoneavg}}{{\xonecopy}}\\
    & \poissmore{{\xtwoavg}}{{\xtwocopy}}\\
    & \poissmore{{\ytwoavg}}{{\ytwocopy}}\\
    &\poissonexpr{\ce{x1x2}} \poissonexpr{\ce{x1y2}}
    \label{simplified_probability}
  \end{split}
\end{equation}
%

turns the 5-dimensional multivariate Poisson into a bivariate distribution,\
only dependent on the amounts of each of the two complexes.
%

For larger system sizes, such as if the volume is increased, the Poisson distribution\
becomes increasingly well approximated by a continuous multivariate Gaussian distribution\
where the mean is equal to the variance \ref{cb3_gaussian}.
%

\smallskip
\begin{equation}
  \begin{split}
    &P(\copynum{x1x2}, \copynum{x1y2}) =\\
    & \gaussmore{{\xoneavg}}{{\xonecopy}}\\
    & \gaussmore{{\xtwoavg}}{{\xtwocopy}}\\
    & \gaussmore{{\ytwoavg}}{{\ytwocopy}}\\
    & \gaussexpr{\ce{x1x2}} \gaussexpr{\ce{x1y2}}
    \label{cb3_gaussian}
  \end{split}
\end{equation}


\subsection*{Results}
\addcontentsline{toc}{section}{Results}

\subsubsection*{The deterministic specificity is a monotonic function of protein concentration}
%\subsubsection*{The deterministic specificity is a monotonic function of \ce{x1} copy number}
%
From Eq. \ref{ss_cb3}, the steady state concentrations of the complexes can be solved numerically\
as a function of ${\conc{x1}}_0$, which can be put into Eq. \ref{sx_def} to obtain the specificity.
%
Given different initial concentrations of \ce{x2} and \ce{y2}, the specificity\
was plotted as a function of the concentration of \ce{x1}.
%
\begin{figure}[H]
  \includegraphics[width=\textwidth]{figures/cb3/deterministic_cb3/deterministic_cb3_conc.png}
  \caption{Plot of the deterministic specificity as a function of ${\conc{x1}}_0$\
    with various different parameters, and V = 10}
  \label{deterministic_cb3}
\end{figure}
%%%probably fix this plot a bit, make the legend labels and x-label more correct

\subsubsection*{The distribution of the stochastic specificity approaches the deterministic mean as system size is increased}
%\subsubsection*{The distribution of the specificity is more continuous as system size increases}
Fig. \ref{landscape_histo} shows how the probability landscape for the two complexes\
compares to the specificity distribution for different system volumes.
%
Concentration is kept constant so copy numbers are increased with the volume.
%
For smaller system sizes, the distribution of the specificity can be seen\
to be not well-behaved.
%
As the volume is increased, the distribution qualitatively becomes more Gaussian, and approaches\
the deterministic solution.
%
This can be seen directly in the probability landscapes, with a discontinuous\
specificity distribution resulting in very discrete probability states, and\
smoother distributions resulting in small distributions centered around the\
deterministic solution in the probability landscape.

\begin{figure}[H]
  \includegraphics[width=\textwidth]{figures/cb3/landscape_to_hist/landscape_to_hist_v2.png}
  \caption{Comparison of the probability landscape to the specificity distribution as volume is changed,\
    from top to bottom, ${\copynum{x1}}_0 = {\copynum{x2}}_0 = {\copynum{y2}}_0 = 10 * V$, V = 1, 10, 100}
  \label{landscape_histo}
\end{figure}
%%%forgot to get rid of the extra x-labels on the landscape

%
%We are interested in the parameter set where the distribution is not so ordered\
%that the behaviour of the distribution is not interesting anymore, but is not so\
%chaotic that the distribution is difficult to analyze.
%


\subsubsection*{The Gaussian approximation of the CB3's specificity and distribution is qualitatively similar to the simulation}
%\subsubsection*{The Gaussian approximation of the system is a good approximation to the behaviour of the simulation}
The probability landscapes of the CB3 motif are consistent with simulation\
when modeled with the chemical master equation or with a Gaussian approximation.

\begin{figure}[H]
  \includegraphics[width=\textwidth]{figures/cb3/comparison_all_same/add_one_is_ok_fixed.png}
  \caption{Comparison of the steady-state probability distribution for the discrete stochastic CB3 model,\
    computed by three different methods.\
    LEFT: numerically computed solution to the Chemical Master Equation (see Methods Eqn. 11).
    CENTER: numerically computed solution using a multivariate Gaussian approximation to the Chemical Master Equation.
    RIGHT: Long simulation using SSA.
    ${\copynum{x2}}_0 = {\copynum{y2}}_0 = 100$, V = 10}
  %\caption{Comparison of the probability landscapes of different methods,\
  \label{landscapes}
\end{figure}
%

In Fig. \ref{ssd_figure}, the log of the sum of squared differences of the probability\
landscape of the Gaussian distribution to the Poisson distribution was calculated\
as a function of volume.
%
As volume is increased, the concentration is kept constant, so the number of\
copies is increased accordingly.

\begin{figure}[H]
  \includegraphics[width=\textwidth]{figures/cb3/ssd/ssd_log.png}
  \caption{Sum of squared differences of the Poisson and Gaussian distribution,\
    ${\copynum{x1}}_0 = {\copynum{x2}}_0 = {\copynum{y2}}_0 = 10 * V$}
  \label{ssd_figure}
\end{figure}
%

This shows, the expected behavior: as system sizes become larger, the Poisson distribution can be better\
approximated by a Gaussian distribution.
%

\begin{figure}[H]
  \includegraphics[width=\textwidth]{figures/cb3/marsaglia_approx_ok/marsaglia_good.png}
  \caption{Marsaglia's approximation to the specificity compared to the simulation data,\
    from left to right, ${\copynum{x2}}_0 = {\copynum{y2}}_0 = 100, 1000$, V = 10, 100}
  \label{marsaglia_good}
\end{figure}

With the Gaussian distribution being a good approximation for the distribution\
of each of the complexes, the distribution of the specificity can be approximated\
by Marsaglia's ratio of normally distributed variables \cite{marsaglia}.
%
A comparison of Marsaglia's approximation to the specificity to the specificity\
obtained from simulation is in Fig. \ref{marsaglia_good}.
%

%% In Fig. \ref{cb3_spec_error}, the mean, \perc{10}, and \perc{90} percentile are\
%% plotted as functions of \ce{x1} copy number.
%% %

%% \begin{figure}[H]
%%   \includegraphics[width=\textwidth]{figures/cb3/spec_error/error_bar_spec_cb3_fixedscale.png}
%%   \caption{Marsaglia's normal approximation with an error bar of the $10^{th}$ and $90^{th}$ percentile,\
%%     ${\copynum{x2}}_0 = {\copynum{y2}}_0 = 100$, V = 10}
%%   \label{cb3_spec_error}
%% \end{figure}

%% Fig. \ref{cb3_spec_error} can be compared to the simulation data plotted\
%% in Fig. \ref{cb3_raw_spec}.


\subsubsection*{The distribution of the stochastic specificity can be approximated as the ratio of two normal random variables}
%\subsubsection*{Changing the denominator to one when it is zero is a safe fix to the specificity}
Because copy numbers are discrete numbers ranging from zero to some maximum value, there can be\
discontinuities in the specificity when the denominator is zero.
%
In simulation data, this is seen at lower \ce{x1} copy numbers\
in Fig. \ref{cb3_raw_spec}, at approximately 20 copy number for the \perc{90}\
percentile, and approximately 10 copy number for the \perc{10} percentile.
%
By changing zero denominators to one, the specificity can always be defined.
%

\begin{figure}[H]
  \includegraphics[width=\textwidth]{figures/cb3/raw_spec/raw_spec2.png}
  \caption{Specificity plot without fixing for infinite values\
    ${\copynum{x2}}_0 = {\copynum{y2}}_0 = 100$, V = 10}
  \label{cb3_raw_spec}
\end{figure}

The data from Fig. \ref{cb3_raw_spec} is plotted again, but with zero denominators\
changed to one in Fig. \ref{add_one}, the deterministic specificity and Marsaglia's\
normal approximation are also plotted for comparison.
%
The median of the simulation data and Marsaglia's approximation match with the\
deterministic solution for higher copy numbers of \ce{x1}.
%
From approximately 50 copy number and lower, both the simulation median and\
Marsaglia median begin to diverge from the deterministic solution.
%
Each of the percentiles (\perc{10}, \perc{50}, \perc{90}) from the simulation match\
Marsaglia's approximation for a wide range of $\copynum{x1}$, and the \perc{10} percentile\
in particular, is very close.
%
%Despite this, changing the denominator has little effect on the probability distribution for\
%systems of large enough size, but, when copy numbers are small enough, the specificty\
%will have enough deviation to be unreliable.
%

\begin{figure}[H]
  \includegraphics[width=\textwidth]{figures/cb3/add_one_is_ok/add_one_spec_ok_v2.png}
  \caption{Comparison between the Gaussian approximation and simulation data with the\
    addition of one to the denominator when it is zero,\
  ${\copynum{x2}}_0 = {\copynum{y2}}_0 = 100$, V = 10}
  \label{add_one}
\end{figure}
%%%maybe i should try not to overload this figure with too much

%%work on this part?

%Because the general trend for the normal approximation to the specificity\
%and the altered simulation data are the same, we can continue with\
%changing the zero denominators for the simulation data.
%
%The only caveat is that care must be taken when looking at data for\
%small copy number regimes, as this area starts to have too many\
%discontinuities to ignore.
%

\subsubsection*{Fluctuations in stochastic specificity at steady-state become significant at low protein copy number}
%\subsubsection*{The median specificity and the width of the distribution are correlated}
For a variety of parameter sets (Fig. \ref{cb3_multiple_spec}), the general behaviour\
of the specificity as a function of the initial copy number of \ce{x1} is the same.
%
The median of the specificity is consistent with the deterministic solution of the\
specificity except for very low $\totalcopy{x1}$, where the effects of having zero copy\
numbers of off-target complex begin to have a large effect on the stochastic specificity.
%
The deterministic solution is a strictly decreasing function of $\totalcopy{x1}$,\
but the median of the solution is a non-monotonic function of $\totalcopy{x1}$.
%
At lower \ce{x1} copy number, the median specificity is lower than the deterministic solution,\
and increases as $\totalcopy{x1}$ increases to match the deterministic solution.
%

\begin{figure}[H]
  \includegraphics[width=\textwidth]{figures/cb3/spec_plots/spec_mult_v3.png}
  \caption{Specificity plotted against \ce{x1} with different parameter sets, and $V=10$}
  \label{cb3_multiple_spec}
\end{figure}


The width of the distribution follows roughly the same trend as the median,\
at higher \ce{x1} copy number, the distribution converges to the deterministic solution,\
and the distribution increases with lower \ce{x1}.
%
When off-target complexes start becoming rare, i.e. the number of off-target complexes is zero,\
the \perc{10} percentile, \perc{90} percentile,\
and median of the specificity become limited by the total amount of \ce{x1}.
%

\begin{figure}[H]
  \includegraphics[width=\textwidth]{figures/cb3/fano_factor/cb3_fano2.png}
  %\includegraphics[width=\textwidth]{figures/cb3/fano_factor/fano_and_prod.png}
  \caption{Fano factor of the complexes plotted against \ce{x1} copy number,\
    ${\copynum{x2}}_0 = {\copynum{y2}}_0 = 100$, V = 10}
  \label{cb3_fano}
\end{figure}

The Fano factor of the complexes, defined as $\sigma^2 / \mu$,\
was plotted against copy number of \ce{x1} in Fig. \ref{cb3_fano}.
%
The noise in the complexes contributes to the specificity's\
distribution.
%
Looking at the distributions of the individual complexes also gives\
consistent results, albeit for a different reason.
%

\subsubsection*{Stochastic specificity shows non-monotonic dependence on protein copy number}
%\subsubsection*{The specificity is a function of the complexes values and distribution}
%\subsubsection*{The behaviour of the specificity is directly linked to the complexes value and distribution}
The behaviour of the specificity can be explained by looking at the\
individual complexes.
%
Because the denominator (\ce{x1y2}) is smaller than the numerator (\ce{x1x2}),\
its deviations have a larger effect on the value of the specificity than the numerator\
for \ce{x1} copy number from 0 to {~}150.
%
For larger \ce{x1}, the amount of complexes is both large and roughly the same,
and so, small changes in the amount of complex have a diminishing effect on the\
distribution of the specificity.
%

\begin{figure}[H]
  \includegraphics[width=\textwidth]{figures/cb3/complex_error/cb3_complex.png}
  \caption{Means of complexes of CB3 with an error bar of $\pm \sigma$,\
    ${\copynum{x2}}_0 = {\copynum{y2}}_0 = 100$, V = 10}
  \label{cb3_complex_error}
\end{figure}

Assuming zero denominators are fixed, the maximum possible value of the\
specificity is $min(\totalcopy{x1}, \totalcopy{x2})$.
%
At very low $\copynum{x1}$, it becomes the case that the amount\
of \ce{x1} becomes limiting, the value of the specificity can not\
exceed $\totalcopy{x1}$, which results in the line seen at lower\
copy numbers of \ce{x1} in Fig. \ref{add_one} and \ref{cb3_multiple_spec}.
%


%%% Local Variables: ***
%%% mode: latex ***
%%% TeX-master: "thesis.tex" ***
%%% End: ***
