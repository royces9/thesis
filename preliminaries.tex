\thesistitle{Specificity in Stochastic Protein-Protein Interaction Networks}

%"Dissertation" for PhD, "Thesis" for master's
\documenttitle{Thesis}

\degreename{Master of Science}

% Use the wording given in the official list of degrees awarded by UCI:
% http://www.rgs.uci.edu/grad/academic/degrees_offered.htm
\degreefield{Chemical and Biomolecular Engineering}

% Your name as it appears on official UCI records.
\authorname{Royce Roland Sato}

% Use the full name of each committee member.
\committeechair{Elizabeth L. Read}
\othercommitteemembers
    {
      Professor Lee Bardwell\\
      Professor Frank Shi
}

\degreeyear{2019}

\copyrightdeclaration
{
  {\copyright} {\Degreeyear} \Authorname
}

% If you have previously published parts of your manuscript, you must list the
% copyright holders; see Section 3.2 of the UCI Thesis and Dissertation Manual.
% Otherwise, this section may be omitted.
% \prepublishedcopyrightdeclaration
% {
% 	Chapter 4 {\copyright} 2003 Springer-Verlag \\
% 	Portion of Chapter 5 {\copyright} 1999 John Wiley \& Sons, Inc. \\
% 	All other materials {\copyright} {\Degreeyear} \Authorname
% }

% The dedication page is optional
% (comment out to exclude).

\acknowledgments {
  First and foremost, I would like to thank my late\
  grandfather for supporting me throughout both\
  my undergraduate and graduate education. Thank you.
  %

  Thank you to Professor Read, and everyone in the\
  Read lab, past and present, I've learned a lot\
  from everyone.
  %

  Thank you to Professor Bardwell and Professor Shi for\
  being on my committee.
  %

  Thank you to Ernest, Erica, Sharnnia, and everyone else from\
  OAI, for both the fellowship and the opportunity to tutor our\
  undergraduate student body.
  %
  
}



% Some custom commands for your list of publications and software.
\newcommand{\mypubentry}[3]{
  \begin{tabular*}{1\textwidth}{@{\extracolsep{\fill}}p{4.5in}r}
    \textbf{#1} & \textbf{#2} \\ 
    \multicolumn{2}{@{\extracolsep{\fill}}p{.95\textwidth}}{#3}\vspace{6pt} \\
  \end{tabular*}
}
\newcommand{\mysoftentry}[3]{
  \begin{tabular*}{1\textwidth}{@{\extracolsep{\fill}}lr}
    \textbf{#1} & \url{#2} \\
    \multicolumn{2}{@{\extracolsep{\fill}}p{.95\textwidth}}
    {\emph{#3}}\vspace{-6pt} \\
  \end{tabular*}
}

% The abstract should not be over 350 words, although that's
% supposedly somewhat of a soft constraint.
\thesisabstract {
  For a cell to properly respond to external stimuli, it must translate the stimulus into\
  a signal from the cell surface, to the inner cell machinery, such as the nucleus.
  %
  This signal transduction occurs by way of protein-protein reaction networks, which act\
  as a set of relays via chemical reactions down to the cell's interior.
  %
  These relays, often called cascades, are seen everywhere in biology, and\
  involve many proteins in intricate reaction networks; moreover, these reaction\
  networks will often have many separate cascades within them, and to complicate\
  things further, cascades can share components with other cascades, opening a\
  possibility for cross-talk.
  %
  The concept of specificity has been used to quantify\
  the relative strength of on-versus off-target interactions for individual protein-protein\
  interactions.
  %
  The concept has also been extended to larger signaling cascades in order to measure
  how much the system prefers the correct pathway over the incorrect pathway.
  In this work, we study specificity in the context of discrete, stochastic chemical kinetics networks. 
  %
  %Despite this cross-talk, cascades are generally well insulated, i.e. a signal\
  %from one cascade will not ``leak'' into another cascade, which must mean there\
  %is an insulating mechanism in-place to stop unnecessary leaking.
  %
  We study two networks motifs, small commonly seen subportions of the network,\
  known as the CB3 and bifan motif.
  %
  We use a variety of stochastic modeling techniques, including Master Equations,\
  simulations, and numerical approximations.
  %
  We find that the fluctuations of specificity in a stochastic network system show\
  qualitatively different behavior than the deterministic, mean-field specificity.
  %
  Such stochastic effects may contribute to cellular optimization of relative protein\
  concentrations, by affecting the balance between strengthening in-pathway signals\
  while minimizing cross-talk.
  %
  %In particular, we look at the stochastic behiavour of the system, and the\
  %specificity of these motifs, a measure of how much the system prefers the\
  %correct pathway over the incorrect pathway.
  %
  
}


%%% Local Variables: ***
%%% mode: latex ***
%%% TeX-master: "thesis.tex" ***
%%% End: ***
