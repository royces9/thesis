\section*{Introduction}
\addcontentsline{toc}{chapter}{Introduction}

%background
In order for a cell to respond to its environment, it must send a signal from\
its surface receptors to the appropriate cell machinery within itself.
%
This is done by using an assortment of chemical signals which cause a\
chain reaction of chemical reactions, known as signaling cascades,\
from which the cell can then interally\
determine the correct response to the given signal.
%
However, what makes these cascades interesting is that oftentimes a portion of a cascade\
can be shared among differing signals.
%
A common example is of bakers yeast, \textit{Saccharomyces cerevicsiae}.
%
Signaling for mating and filamentous invasive growth shares a portion\
of the cascade with eachother \cite{schwartz}.
%
Despite the shared portion of the cascade, yeast very rarely exhibits\
the wrong behaviour when given an external signal.
%
A mating signal does not result in filamentous invasive growth, and a signal for\
invasive growth does not result in mating \cite{ultrasensitive}.
%
But if that is the case, the cell must have a mechanism in place to regulate\
what the signal it received is, and to ignore the intrinsic noise present\
within the cell.
%
How cells achieve robust signal transduction even with shared pathway\
components has been the subject of much study \cite{Rowland5550, sisonadal, vert}.
%
A concept that has been introduced to quantify in-pathway versus out-of-pathway\
interactions (or similarly, on-target versus off-target interactions) has been\
that of \textit{specificity} \cite{ultrasensitive, komarova, mathematical, WALTERMANN2011924}.
%
Previous mathematical definitions and study of specificity in light of particular\
signaling network and insulating mechanisms showed that there are multiple ways
for the cell to regulate and control response levels through external mechanisms\
as well as through the properties of the network itself \cite{mathematical}.
%One way to measure this signal strength is the specificity, defined as the ratio\
%of the concentration or copy number of one species to another.
%
%This value can give a numerical value for how much of the system is invested into\
%one pathway of a cascade over another.

%unknown/problem
While the specificity has been studied before (\cite{ultrasensitive, komarova, mathematical})
%(PLEASE CITE ALL PREV. BARDWELL PAPERS ON SPECIFICITY)
, a thorough analysis of the discrete stochastic distribution of the specificity has not been done, so the influence of fluctuations in these systems is not known.
%
%
%question/purpose
%I deleted this sentence because for the Bifan the ODEs are actually not easily solveable either.
%While the deterministic behaviour of these systems is easily solvable,\
%the behaviour of the distributions of the specificity for discrete, stochastic networks\
%are not known.
%
The purpose of this work was to characterize the behavior of specificity, including fluctuations\
in specificity, in discrete stochastic kinetic networks for two motifs, the CB3 and bifan motif.
%
%experimental approach/results?
This research looks at the intrinsic properties of the network motifs often seen\
in these protein cascades, and how the motif lends to itself the ability to\
have control over noise in the network.
%
We find that the stochastic fluctuations in specificity have in general more complex\
behavior than the mean-field specificity as obtained from deterministic, Ordinary\
Differential Equation kinetics.
%
In some cases, we identified qualitative differences in the behavior of stochastic\
specificity, as compared to the deterministic specificity, as a function of changes\
in protein copy numbers.


%%% Local Variables: ***
%%% mode: latex ***
%%% TeX-master: "thesis.tex" ***
%%% End: ***
